En diferentes campos de la ciencia e ingeniería \cite{pinilla2018coded,fienup1987phase,rivenson2018phase}, la información de fase es imprescindible. Sin embargo, debido a las altas frecuencias de la luz, los sensores únicamente captan la intensidad de la luz incidente en el sensor, perdiendo la información de fase del campo óptico subyacente. En particular, el modelo de propagación que siguen los sistemas ópticos de difracción está dado por $\mathbf{y}_k = \vert \mathbf{A}_k\mathbf{z} \vert^2$ con $k=1,\cdots, K$ proyecciones, donde $\mathbf{z} \in {\mathbb{C}}^{n}$ representa el objeto de interés, $\mathbf{y}_k \in \mathbb{R}^{m}$ son las medidas adquiridas y $\mathbf{A}_k\in \mathbb{C}^{n \times n}$ describe la propagación del frente de onda de la luz hasta el sensor en cada proyección $k$. Cabe destacar que, este sistema de ecuaciones presenta múltiples soluciones, debido a diferentes ambigüedades triviales que resultan de las medidas adquiridas $\mathbf{y}$ para diferentes transformaciones de $\mathbf{z}$ \cite{beinert2015ambiguities,shechtman2015phase}, las cuales se describen a continuación:
\begin{enumerate}
    \item Desplazamiento de la fase global $\mathbf{z}[n] \leftarrow \mathbf{z}[n] \cdot e^{j\Phi_0}$,
    \item Reflexión sobre un eje $\mathbf{z}[n] \leftarrow \overline{\mathbf{z}}[-n]$,
    \item Desplazamiento espacial $\mathbf{z}[n] \leftarrow \mathbf{z}[n + n_0]$,
\end{enumerate}

donde $\overline{(\cdot)}$ corresponde al operador conjugado y $j=\sqrt{-1}$. Para recuperar la información de fase hasta una constante de fase global en imágenes difractivas formadas a partir de sistemas ópticos basados en medidas cuadráticas, la literatura ha incluido un elemento de modulación denominado máscaras de fase, que generan redundancia en las medidas captadas, las cuales se conocen como medidas cuadráticas codificadas. No obstante, el proceso de reconstrucción implica un alto costo computacional para tareas computacionales de inferencia sobre imágenes difractivas. Por consiguiente, algunos trabajos han estudiado la clasificación de objetos usando exclusivamente las medidas adquiridas \cite{bacca2020coupled,douarre2020value,kim2018deep,ziletti2018insightful}, evitando el proceso de reconstrucción de las imágenes. Este tipo de enfoques permiten reducir el tiempo de inferencia de los algoritmos. Sin embargo, estos esquemas de clasificación no se han abordado previamente sobre sistemas ópticos que producen medidas cuadráticas codificadas. Este trabajo propone el diseño de un algoritmo de clasificación de objetos en imágenes difractivas sobre medidas cuadráticas codificadas mediante el uso de aprendizaje profundo. Finalmente, este proyecto se desarrollará con el apoyo del grupo de investigación en diseño de algoritmo y procesamiento de datos multidimensionales (HDSP). El grupo HDSP es experto en el área de procesamiento de imágenes y señales de alta dimensionalidad. Este grupo de investigación se encuentra clasificado actualmente en categoría A1 por Colciencias, adscrito a la Escuela de Ingeniería de Sistemas e Informática de la Universidad Industrial de Santander. 

Pregunta de investigación: ¿Cómo incorporar enfoques de aprendizaje profundo para la clasificación de objetos en sistemas ópticos difractivos basados en medidas cuadráticas codificadas?

\pagebreak