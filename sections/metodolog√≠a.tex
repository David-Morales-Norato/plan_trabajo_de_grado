
Con el fin de alcanzar los objetivos planteados, este trabajo de investigación se llevará acabo mediante las siguientes etapas:

\begin{itemize}
    \item \textbf{Revisión del estado del arte:} 
    La revisión de la literatura es un paso fundamental que se llevará a cabo durante todo el desarrollo del proyecto, con el propósito de estar al tanto de los últimos avances en la literatura. Específicamente, los temas de interés de este proyecto incluyen los conceptos, técnicas y trabajos realizados por otros investigadores, de manera que, se garantice la originalidad del trabajo realizado.
    
    \item \textbf{Modelado matemático del sistema de adquisición:}
    En esta etapa, se desarrollará el modelo matemático para el proceso de adquisición de imágenes difractivas usando medidas cuadráticas codificadas. En particular, el modelo de propagación obtenido incluirá las máscaras de fase que permitan la modulación del campo óptico.
    
    \item \textbf{Implementación del método de clasificación:}
    En este paso se diseñará e implementará un algoritmo de clasificación a partir de medidas cuadráticas codificadas. 
    
    \item \textbf{Simulaciones:}
    En esta etapa, se realizarán las simulaciones del proceso de adquisición de las medidas cuadráticas codificadas, adicionalmente, se entrenará el algoritmo de clasificación propuesto usando las imágenes simuladas. Para evaluar cuantitativamente el rendimiento la tarea de clasificación bajo el algoritmo propuesto, se hará uso de métricas tales como la exactitud, la matriz de confusión y el puntaje F1
    
    \item \textbf{Análisis de resultados}
    Los resultados de la clasificación sobre diferentes conjuntos de datos serán analizados para incluir los posibles cambios al algoritmo propuesto con el objetivo de mejorar la clasificación.

    \item \textbf{Presentación y documentación de resultados:}
    Finalmente, se realizará la documentación de los resultados para su respectiva presentación en el documento final.
    
\end{itemize}

\pagebreak