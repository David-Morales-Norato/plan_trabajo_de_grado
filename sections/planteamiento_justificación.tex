Los algoritmos computacionales basados en aprendizaje profundo han sido ampliamente estudiados en la literatura, especialmente, en sistemas de clasificación de objetos\cite{li2019deep,li2018deep, wang2019development}. En general, los algoritmos de clasificación basados en aprendizaje profundo se llevan a cabo sobre imágenes que se adquieren a través de sistemas ópticos que captan exclusivamente la intensidad de la luz incidente sobre el sensor, omitiendo la información de fase que resulta fundamental en aplicaciones como cristalografía de rayos-x \cite{pinilla2018coded}, astronomía \cite{fienup1987phase}, holografía \cite{rivenson2018phase}, entre otras. La obtención de esta información de fase requiere la implementación de algoritmos computacionales que logren recuperar los datos perdidos. Así que, los algoritmos de recuperación de fase permiten reconstruir la información de un campo óptico inicial con base en la adquisición de medidas de intensidad que siguen un modelo cuadrático de propagación. Múltiples algoritmos de formulación convexa \cite{candes2013phaselift,goldstein2018phasemax} y no convexa \cite{candes2015phase,chen2017solving,wang2017solving,wang2018phase} han sido propuestos para resolver el problema de la recuperación de la fase.

Recientemente, se han desarrollado arquitecturas de redes neuronales para la reconstrucción de imágenes difractivas y diseño de sistemas ópticos que captan mediciones cuadráticas codificadas \cite{cai2020learning}. Estas arquitecturas de redes neuronales incorporan el modelo de adquisición como una capa de la misma arquitectura de red neuronal, permitiendo el diseño de sistemas ópticos donde los pesos entrenables representan las variables optimizables del modelo de propagación. Adicionalmente, algunos trabajos han estudiado la clasificación de objetos usando únicamente medidas de intensidad que siguen un modelo de propagación lineal \cite{bacca2021deep,douarre2020value}, evitando el proceso de reconstrucción de las imágenes, lo cual reduce el tiempo de inferencia en los algoritmos de clasificación. Sin embargo, estos enfoques de clasificación de objetos no se han abordado previamente en el campo de imágenes difractivas codificadas que inducen el problema de recuperación de fase. Por lo tanto, este trabajo de investigación propone el diseño de un algoritmo de clasificación de objetos en imágenes difractivas sobre medidas cuadráticas codificadas mediante el uso de aprendizaje profundo. Finalmente, este proyecto de investigación se desarrollará con el apoyo del grupo de investigación en diseño de algoritmo y procesamiento de datos multidimensionales (HDSP). El grupo HDSP es experto en el área de procesamiento de imágenes y señales de alta dimensionalidad. Este grupo de investigación se encuentra clasificado actualmente en categoría A1 por Colciencias, adscrito a la Escuela de Ingeniería de Sistemas e Informática de la Universidad Industrial de Santander. 

Pregunta de investigación: ¿Cómo incorporar enfoques de aprendizaje profundo para la clasificación de objetos en sistemas ópticos difractivos basados en medidas cuadráticas codificadas?

\pagebreak