%%%%%%%%%%%%%%%%%%%%%%%%%%%%%%
%%        IMPORTS           %%
%%%%%%%%%%%%%%%%%%%%%%%%%%%%%%
\usepackage[spanish]{babel}
\usepackage{subcaption}% Para poder realizar múltiples figuras en una sola
\usepackage{graphicx}% Para poder incluir figuras y las tablas
\usepackage{natbib} % Para poder citar
\usepackage{caption}% Para poder incluir descripciones en las figuras
\usepackage{titlesec}% Para poder modificar los títulos de las secciones
\usepackage{titletoc}% Para poder modificar los títulos en la tabla de contenidos
\usepackage[nottoc,numbib]{tocbibind}% Para incluir la bibliografía en la tabla de contenidos
\usepackage{parskip} % Para poder eliminar la identación en cada párrafo
\usepackage{xcolor} % Para usar textcolor
\usepackage{ulem} % Para tachar texto
\usepackage{float} % Ubicación de figuras
\usepackage{amssymb} % Símbolos matemáticos
\usepackage{amsmath} %  Descripciones y ecuaciones matemáticas
\usepackage{hyperref} % {Para hacer que las referencias sean cliqueables
%%%%%%%%%%%%%%%%%%%%%%%%%%%%%%%%
%% re-definiciones de formato %%
%%%%%%%%%%%%%%%%%%%%%%%%%%%%%%%%


% Modifica los títulos de la tabla de contenidos y de la bibliografía para que concuerden con el formato del plan del proyecto de grado
\addto\captionsspanish
{
    \renewcommand\contentsname{CONTENIDO}%\renewcommand{\contentsname }{\hfill\bfseries TABLA DE CONTENIDO\hfill}%
    \renewcommand{\refname}{Bibliografía}
    \renewcommand\listfigurename{LISTA DE FIGURAS}
    \renewcommand\listtablename{LISTA DE TABLAS}
    \renewcommand\tablename{Tabla}
}

% Modifica lo que tiene que ver con el título de las secciones
\titleformat{\section}{\normalfont\fontfamily{\sfdefault}\fontsize{12}{17}\bfseries}{\thesection}{1em}{}

% Modifica lo que tiene que ver con el título de las subsecciones
\titleformat{\subsection}{\normalfont\fontfamily{\sfdefault}\fontsize{12}{17}\bfseries}{\thesubsection}{1em}{}

% Modifica lo que tiene que ver con el título de las secciones en la tabla de contenidos
\titlecontents{section}[1em]{\normalfont\fontfamily{\sfdefault}\fontsize{12}{17}\bfseries}
{\thecontentslabel\enspace}
{}
{\titlerule*[1pc]{.}\quad\contentspage}[\vskip 4pt]

% Modifica lo que tiene que ver con el título de las subsecciones en la tabla de contenidos
\titlecontents{subsection}[2.7em]{\normalfont\fontfamily{\sfdefault}\fontsize{12}{17}\bfseries}
{\thecontentslabel\enspace}
{}
{\titlerule*[1pc]{.}\quad\contentspage}[\vskip 3pt]

\setlength{\parindent}{0pt} % Elimina la identación en cada párrafo


\renewcommand{\familydefault}{\sfdefault}% Modifica el estilo de letra a una muy parecida a arial 

%% Configura como van a ser las citas
\hypersetup{
    colorlinks=true,
    linkcolor=black,
    filecolor=magenta,      
    urlcolor=cyan,
    citecolor = blue,
}

\DeclareMathOperator*{\minimize}{minimize} % Declaración operador minimize

\DeclareMathOperator*{\maximize}{maximize} % Declaración operador maximize
\DeclareMathOperator*{\subjectto}{subject \ to} % Declaración operador subjectto
\DeclareMathOperator*{\argmax}{arg \ max} % Declaración operador arg max
%%%%%%%%%%%%%%%%%%%%%%%%%%%%%%
%% Información del proyecto %%
%%%%%%%%%%%%%%%%%%%%%%%%%%%%%%

\def\autor{David Santiago Morales Norato}
\def\titulo{Algoritmo de clasificación de objetos en imágenes difractivas basado en medidas cuadráticas codificadas usando un enfoque de aprendizaje profundo}
\def\director{PhD.(c). Andrés Felipe Jerez Ariza}
\def\codirector{Ph.D. Henry Arguello Fuentes}
\def\universidad{Universidad Industrial de Santander}
\def\escuela{Escuela de Ingeniería de Sistemas e Informática}
\def\facultad{Facultad de Ingenierías Fisicomecánicas}
\def\grupo{Grupo de Investigación en Diseño de Algoritmos y Procesamiento de Datos Multidimensionales (HDSP)}
\def\fecha{\today}
\def\codigo{2170102}
